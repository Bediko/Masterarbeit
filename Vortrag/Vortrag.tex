\documentclass[12pt,titlepage]{beamer}
\usepackage{pdfpcnotes}
% language stuff
\usepackage{german}           % deutsche Überschriften etc.
\usepackage[utf8]{inputenc} % direkte Einbgabe von Umlauten

% Layout-Einstellungen
\usepackage{parskip}          % Abstand statt Einrückung
\frenchspacing                % no extra space after periods
\usepackage{parskip}          % paragraph gaps instead of indentation
\usepackage{times}            % default font Times
\tolerance=9000               % avoid words across right border
% miscellaneous<
\usepackage{hhline}           % double lines in tables
\usepackage{amsfonts}         % real numbers etc.
\usepackage[rightcaption]{sidecap} % figure captions on the right (optional)
\usepackage{hyperref}         % for URLs
\usepackage{listings}         % for code samples
 \lstset{breaklines=true,tabsize=1}
\usepackage{etoolbox}
\usepackage{multirow}
\usepackage{tabularx}
\usepackage{amsmath}
\usepackage{multirow}
\usepackage{pbox}
\usepackage{graphicx}
\usepackage[vlined]{algorithm2e}
\usepackage{svg}

\definecolor{HIGHLIGHTCOLOR}{rgb}{0.13333333333333333,0.8156862745098039,0.09019607843137255}



\defbeamertemplate{section page}{mine}[1][]{%
  \begin{centering}
    {\usebeamerfont{section name}\usebeamercolor[fg]{section name}#1}
    \vskip1em\par
    \begin{beamercolorbox}[sep=12pt,center]{part title}
      \usebeamerfont{section title}\insertsection\par
    \end{beamercolorbox}
  \end{centering}
}

\AtBeginSection{\setbeamertemplate{section page}[mine]\frame{\sectionpage}}


\makeatletter
\patchcmd{\insertverticalnavigation}%
{\ifx\beamer@nav@css\beamer@hidetext{\usebeamertemplate{section in sidebar}}\else{\usebeamertemplate{section in sidebar shaded}}\fi}%
{{\usebeamertemplate{section in sidebar}}}{}{}
\makeatother



% Hier bei Bedarf die Seitenränder einstellen
%\geometry{a4paper}
\usetheme{Hannover}
\usecolortheme{rose}
\begin{document}
	\title{Link Prediction}
	\author{Michael Känmmerer}
	\date{\today}
	\institute{}
%------------------------------------------------Folie1----------------------------------------------------------
	\begin{frame}
		\maketitle
	\end{frame}
	\section{Problemstellung}
	\begin{frame}
		\includegraphics[scale=0.4]{img/lpp}
		\pnote{Soziale Netzwerke kompliziert, Vorhersagen insgesamt eher schwierig, für zwei Knoten einfacher und interessant, Beispiel Recommender für neue Freundschaften auf Facebook}
		\pnote{Zwei Graphen, Tranings und Testinterval, Versuche vorauszusagen welche Kanten hinzukommen}
	\end{frame}
	\section{Prädiktoren}
	\begin{frame}
	\frametitle{Prädiktoren}
	\begin{itemize}
	 \item Verschiedene Metriken um einen Score zu berechnen
	 \item Scores werden absteigend sortiert
	 \item größerer Score bedeutet höhere Wahrscheinlichkeit zur Linkbildung
	\end{itemize}
	\end{frame}
	\subsection{Nächste Nachbarn}
	\begin{frame}
	\frametitle{Nächste Nachbarn}
	$$|\Gamma(x) \cap \Gamma(y)|$$
	\includegraphics[scale=0.3]{img/nn}
	\end{frame}
	\subsection{Jaccard-Koeffizienten}
	\begin{frame}
	\frametitle{Jaccard-Koeffizienten}
	\begin{itemize}
		\item Nächste Nachbarn sind nicht normalisiert
		\item Definiere Wahrscheinlichkeit gemeinsamen Nachbarn zu ziehen wenn aus Nachbarn von x und y gezogen wird
		\item Mehr gemeinsame Nachbarn ergeben bessere Bewertung
		\item Schlechter als Nächste Nachbarn
		
		$$J(x,y)=\frac{|\Gamma(x) \cap \Gamma(y)|}{|\Gamma(x) \cup \Gamma(y)|}$$
		\pnote{Anzahl der gemeinsamen Nachbarn von zwei Knoten}
		\pnote{Bild:E und C haben gemeinsame Nachbarn, also verbindung wahrscheinlich}
		
	\end{itemize}
	\end{frame}
	\subsection{Adamic und Adar}
	\begin{frame}
	\frametitle{Adamic und Adar}
	\begin{itemize}
	\item Misst gemeinsame Nachbarn von x und y
	\item Gewichtet kleinere Abweichungen stärker
	\item Besser als Jaccard und Nächste Nachbarn
	\end{itemize}
	
	$$A/A(x,y) = \sum_{z \in \Gamma(x) \cap \Gamma (y)} \frac{1}{\log | \Gamma(z)|}$$
	
	\end{frame}
	\subsection{Kürzeste Pfade}
	\begin{frame}
	\frametitle{Kürzeste Pfade}
	\includegraphics[scale=0.3]{img/kp}
	\pnote{Einfachstes Pfadbasierte verfahren}
	\pnote{Je kürzer der Pfad zwischen zwie Knoten desto höhere wahrscheinlichkeit}
	\end{frame}
	\subsection{Katz}
	\begin{frame}
	\frametitle{Katz}
	\begin{itemize}
	\item Berücksichtigt alle Pfade zwischen x und y
	\item Gewichtet alle Pfade einer länge l
	\item Lange Pfade werden schwächer gewichtet
	\item besser als kürzeste Pfade
	\end{itemize}
	$$k(x,y)= \sum_{l=1}^{\infty} \beta^l \cdot |\mbox{paths}_{x,y}^{\langle l \rangle}|$$
	\pnote{Beta: Steuert Verhalten, kleines Beta macht ähnlicher zu nächste Nachbarn}
	\pnote{Zweiter Term: Menge der Pfade mit länge l von x nach y}
	\end{frame}
	\subsection{Hitting Time}
	\begin{frame}
	\frametitle{Hitting Time}
	\begin{itemize}
	\item Beschreibt Anzahl Schritte die ein Random Walk von x nach y benötigt
	\item Kleinere Hitting Time je näher die Knoten beieinander liegen
	\item Wird symmetrisch wenn Hin- und Rückweg gemessen werden
	\end{itemize}
	$$C_{x,y}=H_{x,y} + H_{y,x}$$
	\end{frame}
	
	\begin{frame}
	\frametitle{Hitting Time}
	\begin{itemize}
	\item hohe stationäre Wahrscheinlichkeiten sind problematisch
	\item kann durch Resetwahrscheinlichkeit gemindert werden
	\item bei großen Netzen wird mit stationärer Wahrscheinlichkeit normiert
	\end{itemize}
	$$NHH(x,y)= H_{x,y}\cdot \pi_y + H_{y,x} \cdot \pi_x$$
	\end{frame}
	\begin{frame}
	\frametitle{Hitting Time}
	\includegraphics[scale=0.4]{img/hhp}
	\pnote{ z hohe stationäre wahrscheinlichkeit da oft anspringbar}
	\end{frame}
	\subsection{Rooted Pagerank}
	\begin{frame}{Rooted Pagerank}
	\begin{itemize}
	\item Misst die stationären Wahrscheinlichkeiten von x und y
	\item Wird symmetrisch durch vertauschen von x und y
	\item Kann mit Hilfe einer Degree- und einer Adjazenzmatrix für alle paare bestimmt werden 
	\end{itemize}

	$$RPR = (1-\beta)(I-\beta N)^{-1}$$
	\pnote{beta: Wahrscheinlicheit von x zu y zu springen}
	\pnote{1-beta: zu zufälligem knoten springen}
	\pnote{degree-matrix: Diagonalmatrix wo Anzahl der ausgehenden Kanten eingetreagen sind}	
	\end{frame}
	\subsection{Simrank}
	
	\begin{frame}{SimRank}
	\begin{itemize}
	\item Idee dass ähnliche Knoten über ähnliche Nachbarn definiert sind
	\item Bei random Walk ist Simrank $\gamma^l$
	\item l ist Zufallsvariable wann sich zwei Random Walks treffen die von x und y starten
	\end{itemize}
	
	$$similarity(x,y) = \gamma \cdot \frac{\sum_{a\in \Gamma(x)}\sum_{b \in\Gamma(y)} similarity(a,b)}{|\Gamma(x)| \cdot |\Gamma(y)|}
$$
	
	\end{frame}
	\section{Experimentelle Überprüfung}
	\subsection{Aufbau}
	\begin{frame}{Aufbau}
	\begin{itemize}
	\item Es wurden Netzwerke aus gemeinsamen Autorenschaften verschiedener Physikfelder von Arxive.com generiert
	\item Es wurden Zeiträume für eine Trainungs- und eine Testphase bestimmt
	\item Autoren mit mehr als drei Papern in beiden Phasen werden als Core bezeichnet 
	\end{itemize}
	\end{frame}
	\begin{frame}{Aufbau}
	\flushleft
	\includegraphics[scale=0.4]{img/fig1}
	\end{frame}
	\subsection{Ergebnisse}
	\begin{frame}{Ergebnisse}
	\begin{itemize}
	\item Baseline ist ein Zufallsprädiktor
	\item $0,15\%$ bis $0,48\%$ sind zufällig richtig.
	\item Da alle Prädiktoren besser sind wurden noch Nächste Nachbarn und kürzeste Wege verglichen
	\end{itemize}
	\end{frame}
	\begin{frame}{Ergebnisse}
	\includegraphics[scale=0.35]{img/fig3}
	\end{frame}
	\begin{frame}{Ergebnisse}
	\includegraphics[scale=0.35]{img/fig4}
	\end{frame}
	\begin{frame}{Ergebnisse}
	\includegraphics[scale=0.35]{img/fig5}
	\end{frame}
	\subsection{Small World Problem}
	\begin{frame}{Small World Problem}
	\begin{itemize}
	\item Die kürzesten Pfade sind eigentlich unbrauchbar
	\item Community stark vernetzt, sodass virtuell jeder über einen kurzen Pfad mit jedem verbunden ist
	\item Schlecht für die Link prediction, da so jedes Knotenpaar hohe Wahrscheinlichkeit bekommt
	\end{itemize}
	
	\end{frame}
	\subsection{Reduzierung auf 3er Distanzen}
	\begin{frame}
	\begin{itemize}
	\item Löst das Small World Problem, da kurze Pfade entfernt werden
	\item Nächste Nachbarn exisiteren nicht mehr
	\item übrig gebliebene Prädiktoren liefern ähnlich gute Ergebnisse
	\end{itemize}
	
	\end{frame}
	\section{Fazit}
	\begin{frame}
	\begin{itemize}
	\item Bester Prädiktor liegt nur in $16\%$ richtig
	\item Viele Informationen nicht im Netz kodiert
	\item Neue und alte Kollaborationen unterschiedlich gewichten
	\item Geographische Zusammenhänge Berücksichtigen
	\end{itemize}
	
	\end{frame}
	
	
	
	
	
	

	


	

\end{document}










